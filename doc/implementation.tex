%%%%%%%%%%%%%%%%%%%%%%%%%%%%%%%%%%%%%%%%%%%%%%%%%%%%%%
% Filename:      implementation.tex
% Author:        Junwei Wang(wakemecn@gmail.com)
% Last Modified: 2012-05-02 11:12
% Description:
%%%%%%%%%%%%%%%%%%%%%%%%%%%%%%%%%%%%%%%%%%%%%%%%%%%%%%
\chapter{实现}
在本章中我们将要讨论实现第三章构造的一些实际问题,包括一些优化,我们所开发的工具包
的描述以及性能的测量。
\section{解密算法的效率改进}
尽管没有必要通过新的技术减少setup、密钥生成和加密算法的群运算,但是这却可以大幅提升
解密算法的性能。我们将在次介绍这些改进,并在4.3节中给出他们的作用。\par
\vspace{5mm}
\textbf{优化解密策略}\quad 第三章出的递归算法中,被对私钥属性匹配的叶子节点进行两次
配对,对沿着节点到根的每个内部节点(不包括根节点)至多一次指数运算。递归部分的最后一
步会有一次额外的配对。当然,每个阈值为$k$的内部节点,只保留他的$k$个孩子。考虑到在
提前叶节点满足和选择可以满足整个访问树的自己的时间,我们可能避免估计DecryptNode,因为
结果最终不被使用。\par
更精确的讲,设$M$是访问树$\mathcal{T}$的节点的子集。我们定义函数$\mathbf{restrict}(
\mathcal{T},M)$是从$\mathcal{T}$删除以下节点(同时保持阈值不变)形成的访问树。
首先,我们删除所有不在$M$中的节点,下一步我们删除沿着孩子数小于阈值$k_x$的内部节点
$x$一直到$\mathcal{T}$的根节点的所有节点。重复上述步骤直至没有节点可以删除,这时侯
得到的就是$\mathbf{restrict}(\mathcal{T},M)$。因此给定访问树$\mathcal{T}$和满足访问树
的属性集$\gamma$,自然的问题是选择集合$M$使得$\gamma$满足$\mathbf{restrict}(
\mathcal{T},M)$并且$M$中的叶子数目是最少的(考虑到配对操作的代价是最昂贵的)。这可以通过对树的一次遍历的递归算法很容易的实现。算法DecryptNode在
$\mathbf{restrict}(\mathcal{T},M)$上会取得与原来相同的结构。\par
\vspace{5mm}
\textbf{直接计算}\quad
进一步性能改进,可以放弃DecryptNode函数并采取更直接的计算。直观的讲,我们想象展平
树上对DecrypNode的递归调用,然后合并指数运算到每一个(用过的)叶子节点上。更精确的讲,
令$\mathcal{T}$是根为$r$的访问树,$\gamma$是属性集合,$M \subseteq \mathcal{T}$使得
$\gamma$满足$\mathbf{restrict}(\mathcal{T},M)$。同时假设$M$是最小的,即没有内部节点的
孩子数比阈值大。设$L \subseteq M$是$M$中的叶子节点。那么对于每一个$l \in L$,
$l$到$r$的路径表示为
$$\rho(l)=(l,\mathbf{parent}(l),\mathbf{parent}(\mathbf{parent}(l)),...r) .$$
同时,

