%%%%%%%%%%%%%%%%%%%%%%%%%%%%%%%%%%%%%%%%%%%%%%%%%%%%%%
% Filename:      abstract.tex
% Author:        Junwei Wang(wakemecn@gmail.com)
% Last Modified: 2012-04-15 14:44
% Description:
%%%%%%%%%%%%%%%%%%%%%%%%%%%%%%%%%%%%%%%%%%%%%%%%%%%%%%
\chapter*{摘\qquad 要}
在分布式系统中,只有当用户拥有某些凭据或者属性才能够访问数据。
目前,执行这种策略的唯一方式是采用一个可信的服务器来存储数据
并调整访问控制。然而,如果任一存储数据的服务器被攻破,那么数
据的保密性就会大打折扣。在本文中,我们提供一个称之为密文策略
的基于属性的加密来实现对加密数据的复杂的访问控制。通过我们的
技术,加密数据甚至可以存储在不可信的存储服务器上。此外,我们
的方法可以抵制合谋攻击。先前的基于属性的加密方案用属性来描述
被加密的数据,而将策略写入用户的的密钥里;然而在我们的系统中
属性用来描述用户的凭据,加密数据的一方来决定谁可以解密的策略
。因此,我们的加密方案在概念上更接近与传统的访问控制方法,如
基于角色的访问控制(Role-Base Access Control,RBAC)。\par
除此之外,我们提供了我们系统的实现和性能测试。
\addcontentsline{toc}{section}{摘\qquad 要}
\chapter*{ABSTRACT}
In several distributed systems a user should only be %
able to access data if a user possesses a certain set of %
credentials or attributes. Currently, the only method %
for enforcing such policies is to employ a trusted server %
to store the data and mediate access control. However, %
if any server storing the data is compromised, then the %
confidentiality of the data will be compromised. In this %
paper we present a system for realizing complex access %
control on encrypted data that we call Ciphertext-Policy %
Attribute-Based Encryption. By using our techniques %
encrypted data can be kept confidential even if the %
storage server is untrusted; moreover, our methods are %
secure against collusion attacks. Previous Attribute-Based %
Encryption systems used attributes to describe %
the encrypted data and built policies into user’s keys;%
while in our system attributes are used to describe a %
user’s credentials, and a party encrypting data determines %
a policy for who can decrypt. Thus, our methods %
are conceptually closer to traditional access control %
methods such as Role-Based Access Control (RBAC).\par
In addition, we provide an implementation of our system %
and give performance measurements.
\addcontentsline{toc}{section}{ABSTRACT}
