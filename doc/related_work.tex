%%%%%%%%%%%%%%%%%%%%%%%%%%%%%%%%%%%%%%%%%%%%%%%%%%%%%%
% Filename:      related_work.tex
% Author:        Junwei Wang(wakemecn@gmail.com)
% Last Modified: 2012-04-17 10:13
% Description:
%%%%%%%%%%%%%%%%%%%%%%%%%%%%%%%%%%%%%%%%%%%%%%%%%%%%%%
%\chapter{相关工作}
%\section{IBE及其扩展}
%Sahai和Waters提出了ABE的概念的同时,提出了一个叫做模糊的基于身份的加密
%\footnote{Fuzzy-IBE,简称FIBE}的特别系统。FIBE机制建立在一些IBE思想的基础
%\cite{BF:IBE,Shamir:IBE,Cocks:IBE,CHK:fspk,BB:IBE}之上。在FIBE中,身份被
%看作是属性的集合。如果私钥拥有身份$\omega$,密文基于身份$\omega'$,当且仅
%当$\omega$和$\omega'$“设置重叠”的距离度量足够接近私钥才能解密密文。换言之,
%如果消息用属性集合$\omega'$加密,私钥由属性集合$\omega$构成,当且仅当
%$|\omega\cap\omega'|\geq d$\footnote{d在加密是已经确定}时,私钥才能解密密
%文。因此,FIBE实现了容错并适用于生物身份识别。然而,FIBE却不适用我们本文
%的主要目的即数据的访问控制。由于FIBE的主要目标是实现容错,因此它唯一支持
%的访问结构是阈值在加密时已经确定的阈值门。\par
%
