%%%%%%%%%%%%%%%%%%%%%%%%%%%%%%%%%%%%%%%%%%%%%%%%%%%%%%
% Filename:      construction.tex
% Author:        Junwei Wang(wakemecn@gmail.com)
% Last Modified: 2012-04-23 11:56
% Description:
%%%%%%%%%%%%%%%%%%%%%%%%%%%%%%%%%%%%%%%%%%%%%%%%%%%%%%
\chapter{CP-ABE构造}
本章中我们提供了BSW方案的构造。我们首先介绍描述密文的访问树和描述私钥的属性的模型,
其次我们给出我们方案的描述,最后我们给出安全性,高效性的讨论。
%%%%%%%%%%%%%%%%%%%%%%%%%%%%%%%%%%%%%%%%%%%%%%%%%%%%%%
%%%%%%%%%%%%%%%%%%%%%%%%%%%%%%%%%%%%%%%%%%%%%%%%%%%%%%
\section{模型}
在我们的构造中私钥被描述性的属性集合$S$所确定。一方想解密数据必须拥有满足访问树结构的
私钥。\par
访问树中的每个内部节点是一个阈值门,叶子节点与属性关联。(请注意,这样的描述是非常
具有表达力的。例如,我们分别用2 of 2和1 of 2的阈值门来表达树中“AND”和“OR”门。)当且
仅当私钥的属性分配到树的节点并使树能满足,持有私钥的用户将能够解密密文。尽管在我们的
方案里,属性用来标记私钥,我们还是用与\cite{GPSW:ABE}相同的概念来描述访问树。\par
\vspace{5mm}
\noindent\textbf{访问树$\mathcal{T}$}\quad 用$\mathcal{T}$代表访问结构,用通过他的孩
子和阈值描述阈值门来代表树的每个非叶子节点。设节点$x$的孩子数目为$num_x$,阈值为$k_x$,
那么$0<k_x\leq num_x$。当$k_x=1$,阈值门是或门;当$k_x=num_x$,阈值门是与门。树的每个
叶子节点$x$用一个属性和阈值$k_x=1$描述。\par
为了方便的使用访问树,我们定义了一些函数。函数$\textbf{parent}(x)$表示节点$x$的
父节点,仅当$x$是叶子节点时,函数$\textbf{att}(x)$被定义为$x$所描述的属性。访问树
$\mathcal{T}$同时定义了每个节点的子节点的被标记为从1到$num$的顺序索引值,函数
$\textbf{index}(x)$定义为与节点$x$关联的这样一个索引值,且对一个给定的密钥,索引值
以任意的方式唯一的赋值给访问结构中的节点。\par
\vspace{5mm}
\noindent\textbf{满足访问树}\quad 令$\mathcal{T}$是根为$r$的访问树,$\mathcal{T}_x$
表示以$x$为根的$\mathcal{T}$的子树,因此$\mathcal{T}$等同于$\mathcal{T}_r$。如果属性
集合$\gamma$满足访问树$\mathcal{T}_x$,我们表示为$\mathcal{T}_x(\gamma)=1$。我们用如
下递归的方式计算$\mathcal{T}_x(\gamma)$:如果$x$是非叶子节点,对于$x$的所有孩子,
计算$\mathcal{T}_{x'}(r)$,当且仅当至少$k_x$个孩子返回1,$\mathcal{T}_x(\gamma)$返回1;
如果$x$是叶子节点,那么当且仅当$\textbf{attr}(x)\in \gamma$时,$\mathcal{T}_x(\gamma)$
返回1。
%%%%%%%%%%%%%%%%%%%%%%%%%%%%%%%%%%%%%%%%%%%%%%%%%%%%%%
%%%%%%%%%%%%%%%%%%%%%%%%%%%%%%%%%%%%%%%%%%%%%%%%%%%%%%
\section{构造}
设$\mathbb{G}_0$是阶为素数$p$的双线性映射,$g$是$\mathbb{G}_0$的生成元,
$e:\mathbb{G}_0\times \mathbb{G}_0\to\mathbb{G}_1$是双线性映射。安全参数$\kappa$决定
群的规模。同时,我们定义了拉格朗日系数$\Delta_{i,S}$,$i\in\mathbb{Z}_p$,$S$是集合
$\mathbb{Z}_p$中的元素:$\Delta_{i,S}(x)=\prod_{j\in S,j\neq i}\frac{x-j}{i-j}$。
哈希函数$H:{0,1}^*\to \mathbb{G}_0$是random oracle,这个函数可以把任意用二进制串
描述的的属性映射到随机的群元素。我们的构造如下。\par
%%%%%%%%%%%%%%%%%%%%%%%%%%%%
\vspace{5mm}
\noindent\textbf{Setup}\quad 
Setup算法选择生成元为$g$,阶为素数$p$的双线性映射
$\mathbb{G}_0$,指数$\alpha,\beta\in\mathbb{Z}_p$。生成公钥
$$\textrm{PK} = \mathbb{G}_0,g,h=g^{\beta},f=g^{1/\beta},e(g,g)^{\alpha}$$
和主密钥$\textrm{MK} = \beta,g^\alpha$。(注意$f$仅用于委托。)\par
%%%%%%%%%%%%%%%%%%%%%%%%%%%%
\vspace{5mm}
\noindent\textbf{Encryption(PK,$M$,$\mathcal{T}$)}\quad
加密算法在树访问结构$\mathcal{T}$下加密消息$M$。算法首先为$\mathcal{T}$每个节点$x$
(包括叶子节点)选择一个多项式$q_x$。从树的根节点$R$开始,自上而下选择多项式。节点
$x$的多项式$q_x$的度$d_x$比该节点的阈值$k_x$少一,即$d_x=k_x-1$。\par
算法从根节点$R$开始选择随机数$s\in\mathbb{Z}_p$,并设置$q_R(0)=s$。然后,算法随机
选择多项式$q_R$上的$d_R$个点来完全定义$q_R$。对于其他的顶点$x$,令
$q_x(0)=q_{parent(x)}(index(x))$,随机选择其它$d_x$个顶点来完全定义$q_x$。\par
设$\mathcal{T}$中所有叶子节点的集合为$Y$,那么在给定的树形访问结构$\mathcal{T}$下计算
密文
\begin{equation*}
\begin{split}
\textrm{CT}=&(\mathcal{T},\widetilde{C}=Me(g,g)^{\alpha s},C=h^s,\\
&\forall y\in Y:\quad C_y=g^{q_y(0)},C_y'=H(\textbf{att}(y))^{q_y(0)}).
\end{split}
\end{equation*}
%%%%%%%%%%%%%%%%%%%%%%%%%%%%
\vspace{5mm}
\noindent\textbf{KeyGen(MK,$S$)}\quad
密钥生成算法的输入是属性集合$S$,输出为被$S$所标记的密钥。算法首先选择随机数
$r\in \mathcal{Z}_p$,然后对每一个$j\in S$选择随机数$r_j\in \mathcal{Z}_p$。最后计算出
私钥
\begin{equation*}
\begin{split}
\textrm{SK}=&(D=g^{(\alpha+\gamma)/\beta},\\
&\forall j\in S:\quad D_j=g^r\cdot H(k)^{rj},D_j'=g^{rj}).
\end{split}
\end{equation*}
%%%%%%%%%%%%%%%%%%%%%%%%%%%%
\vspace{5mm}
\noindent\textbf{Decrypt(CT,SK)}\quad
我们的解密算法是一个递归算法。为了便于讨论,我们提出了解密算法的最简单形式,并在下一小
节中提出了潜在的性能改进。\par
我们首先定义递归算法Decrypt(CT,SK,$x$),它用密文
$\textrm{CT}=(\mathcal{T},\widetilde{C},C,\forall y\in Y:\quad C_y,C_y')$,与属性集合
$S$关联的私钥SK,$\mathcal{T}$中的节点$x$作为输入。\par
当节点$x$是叶子节点,令$i=\textbf{att}(x)$,如果$i\in S$,那么
\begin{equation*}
\begin{split}
DecryptNode(CT,SK,x) &= \frac{e(D_i,C_x)}{e(D_i',C_x')}\\
&=\frac{e(g^r\cdot H(i)^{r_i},g^{q_x(0)})}{e(g^{r_i},H(i)^{q_x(0)})}\\
&=e(g,g)^{rq_x(0)}.
\end{split}
\end{equation*}
如果$i\notin S$,那么DecryptNode(CT,SK,$x$)$=\perp$。\par
现在我们考虑$x$是非叶子节点时的递归情况。算法DecryptNode(CT,SK,$x$)工作方式如下:
对于$x$的所有字节点$z$,计算$F_z=\textrm{DecryptNode(CT,SK,}z\textrm{)}$。
令$S_x$为$k_x$大小的满足$F_z\neq \perp$的子节点$z$的集合。如果不存在这样的集合,
那么这个节点不满足,且函数返回$\perp$。\par
否则,我们计算
\begin{equation*}
\begin{split}
F_x &=\prod_{z\in S_x}F_x^{\Delta_{i,S_x'(0)}}
\quad \text{,其中,$i=index(z),S_x'=\{index(z),z\in S_x\}$}\\
&=\prod_{z\in S_x}(e(g,g)^{r\cdot q_z(0)})^{\Delta_{i,S_x'(0)}}\\
&=\prod_{z\in S_x}(e(g,g)^{r\cdot q_{parent(z)}(index(z))})^{\Delta_{i,S_x'(0)}}\\
&=\prod_{z\in S_x}e(g,g)^{r\cdot q_z(0)\cdot \Delta_{i,S_x'(0)}}\\
&=e(g,g)^{r\cdot q_x(0)} \qquad \text{(使用多项式插值)}
\end{split}
\end{equation*}
并返回结果。\par
在定义了DecryptNode函数之后,我们定义解密算法。算法首先调用Decrypt(CT,SK,$R$),
$R$是树$\mathcal{T}$的根节点。如果树满足$S$,我们令
$$A=\textrm{DecryptNode(CT,SK,}R\textrm{)}=e(g,g)^{rq_R(0)}=e(g,g)^{rs}$$.
现在算法通过下面计算解密
$$\widetilde{C}/(\frac{e(C,D)}{A})=
\widetilde{C}/(\frac{e(h^s,g^{(\alpha+\gamma)/\beta})}{e(g,g)^{rs}})=M$$









